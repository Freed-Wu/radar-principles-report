\documentclass[../main]{subfiles}
\begin{document}

\chapter{单目标测速实验}%
\label{cha:velocity}

\section{实验要求}%
\label{sec:requirement}

\begin{itemize}
  \item 掌握脉冲多普勒雷达测量目标移动速度的基本原理。
  \item 了解雷达测速基本原理,通过实际操作掌握相关仪器仪表使用方法。
  \item 了解雷达系统信号测量目标距离的软硬件条件及具体实现方法。
\end{itemize}

\section{实验方法}%
\label{sec:\arabic{chapter}method}

\subsection{实验时间}%
\label{sub:\arabic{chapter}time}

9:50--10:15。

\subsection{人员分工}%
\label{sub:\arabic{chapter}people}

\begin{table}[htbp]
  \centering
  \caption{人员分工}%
  \label{tab:\arabic{chapter}people}
  \tiny
  \csvautobooktabular{tab/\arabic{chapter}people.csv}
\end{table}

\subsection{实验过程}%
\label{sub:\arabic{chapter}process}

\begin{enumerate}
  \item 实验准备,通过博士能 ELITE 激光测距仪测量,并使用白色粉笔标定距离。
  \item 实验开始,指挥的同学明确各人职责,各单位负责人员就位,远距离通讯组同
    学建立电话联系。
  \item 打开电脑与雷达,各单位听从总指挥,首先进行一次流程测试,无误后开始正
    式实验。
  \item 指挥示意开始,目标同学首先以较慢速度往返运动。
  \item 手动计时同学记录目标同学的运动时间,除以距离算出平均速度,同时雷达进
    行测速,负责数据记录的同学对测得结果拍照记录。
  \item 目标同学逐次加快运动速度,或骑自行车,或有卡车经过,重复上述过程。
\end{enumerate}

\subsection{数据记录}%
\label{sub:\arabic{chapter}record}

\begin{table}[htbp]
  \centering
  \caption{数据记录}%
  \label{tab:\arabic{chapter}record}
  \tiny
  \csvautobooktabular[respect percent]{tab/\arabic{chapter}record.csv}
\end{table}

\section{实验分析}%
\label{sec:\arabic{chapter}analysis}

误差如图~\ref{fig:\arabic{chapter}record}。

\begin{figure}[htbp]
  \centering
  \includegraphics[
    width = 0.8\linewidth,
  ]{\arabic{chapter}record}
  \caption{速度误差}%
  \label{fig:\arabic{chapter}record}
\end{figure}

\section{实验讨论}%
\label{sec:\arabic{chapter}discuss}

误差原因主要如下:

\begin{itemize}
  \item 雷达本身的系统误差,已经发现有些时候速度显示数据只会显示几个固定的常
    数,不能连续变化;
  \item 计时的随机误差;
  \item 测距的随机误差;
  \item 志愿者不能保持匀速直线运动;
  \item 环境噪声;
  \item 实验数据读取的粗大误差。尽管我希望用电脑打字的方式记录数据,但实验记
    录人员坚持要用笔纸记录数据,理由是实验报告最后要附上记录数据的纸的照片。
    结果实验结束后他将照片上传到群后才发现字迹根本无法有效辨识,尤其是7和1难
    以区分。尽管我已经很努力地辨认了,但仍然不敢保证能全部识别正确。
\end{itemize}

\section{实验小结}%
\label{sec:\arabic{chapter}conclusion}

针对误差,解决方案如下:

\begin{itemize}
  \item 修好雷达;
  \item 换用更精确的计时仪器;
  \item 换用更精确的测距仪器;
  \item 选择能匀速直线运动的设备代替志愿者;
  \item 选择环境噪声更小的环境进行实验;
  \item 以后实验采取电子记录的方式。
\end{itemize}

\end{document}

