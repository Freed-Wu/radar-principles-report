\documentclass[../main]{subfiles}
\begin{document}

\chapter{单目标测距实验}%
\label{cha:distance}

\section{实验要求}%
\label{sec:requirement}

\begin{itemize}
  \item 掌握脉冲多普勒雷达测量目标距离的基本原理。
  \item 了解雷达测距的基本原理,通过实际操作掌握雷达系统的使用方法。
\end{itemize}

\section{实验方法}%
\label{sec:\arabic{chapter}method}

\subsection{实验时间}%
\label{sub:\arabic{chapter}time}

10:26--10:38。

\subsection{人员分工}%
\label{sub:\arabic{chapter}people}

\begin{table}[htbp]
  \centering
  \caption{人员分工}%
  \label{tab:\arabic{chapter}people}
  \tiny
  \csvautobooktabular{tab/\arabic{chapter}people.csv}
\end{table}

\subsection{实验过程}%
\label{sub:\arabic{chapter}process}

\begin{enumerate}
  \item 实验开始,指挥的同学明确各人职责,各单位负责人员就位,远距离通讯组同学建立电话联系。
  \item 打开电脑与雷达,各单位听从总指挥,首先进行一次流程测试,无误后开始正式实验。
  \item 指挥示意开始,目标同学在指定位置处小幅度前后晃动。
  \item 两侧通讯人员沟通后,雷达处记录同学记录目标志愿者实际所在位置和雷达测得距离数据,并对测得结果拍照记录。
  \item 目标同学更换位置前后晃动,重复上述过程。
  \item 一名同学负责及时对比测量值与实际值,剔除严重不符合实际的数据组。
\end{enumerate}

\subsection{数据记录}%
\label{sub:\arabic{chapter}record}

\begin{table}[htbp]
  \centering
  \caption{数据记录}%
  \label{tab:\arabic{chapter}record}
  \csvautobooktabular[respect percent]{tab/\arabic{chapter}record.csv}
\end{table}

\section{实验分析}%
\label{sec:\arabic{chapter}analysis}

误差如图~\ref{fig:\arabic{chapter}record}。

\begin{figure}[htbp]
  \centering
  \includegraphics[
    width = 0.8\linewidth,
  ]{\arabic{chapter}record}
  \caption{距离误差}%
  \label{fig:\arabic{chapter}record}
\end{figure}

\section{实验讨论}%
\label{sec:\arabic{chapter}discuss}

误差原因主要如下:

\begin{itemize}
  \item 雷达精度不够高;
  \item 测距的随机误差;
  \item 志愿者不能保持在原地切向运动;
  \item 环境噪声;
  \item 距离过远时应考虑地球曲率和大气折射的影响;
  \item 实验数据读取的粗大误差。最后一组数据明显不对。
\end{itemize}

\section{实验小结}%
\label{sec:\arabic{chapter}conclusion}

针对误差,解决方案如下:

\begin{itemize}
  \item 换精度更高的雷达;
  \item 换用更精确的测距仪器;
  \item 选择能在原地切向运动的设备代替志愿者;
  \item 选择环境噪声更小的环境进行实验;
  \item 本实验距离不够远,修正意义不大;
  \item 采取 Tauta 法则剔除粗大误差。
\end{itemize}

\end{document}

