\documentclass[../main]{subfiles}
\begin{document}

\chapter{雷达录取数据分析实验}%
\label{cha:analysis}

\section{实验要求}%
\label{sec:requirement}

\begin{itemize}
  \item 掌握脉冲多普勒雷达测量目标移动速度与测量目标距离的基本原理。
  \item 掌握雷达录取数据的上位机软件的使用方法,将雷达录取数据记录于电脑。
  \item 学会使用 Matlab 软件对雷达录取数据进行分析和处理。
\end{itemize}

\section{实验方法}%
\label{sec:\arabic{chapter}method}

\subsection{实验时间}%
\label{sub:\arabic{chapter}time}

9:50--10:38。

\subsection{人员分工}%
\label{sub:\arabic{chapter}people}

\begin{table}[htbp]
  \centering
  \caption{人员分工}%
  \label{tab:\arabic{chapter}people}
  \tiny
  \csvautobooktabular{tab/\arabic{chapter}people.csv}
\end{table}

\subsection{实验过程}%
\label{sub:\arabic{chapter}process}

\begin{enumerate}
  \item 实验开始,指挥的同学明确各人职责,各单位负责人员就位,远距离通讯组同
    学建立电话联系。
  \item 打开电脑与雷达,各单位听从总指挥,首先进行一次流程测试,无误后开始正
    式实验。
  \item 指挥示意开始,目标同学首先以较慢速度往返运动进行雷达测速实验。
  \item 手动计时同学记录目标同学的运动时间,除以距离算出平均速度,同时雷达进
    行测速,负责计算机操控的同学对雷达录取结果保存、重新命名。
  \item 目标同学逐次加快运动速度,重复上述过程。
  \item 一名同学负责及时对比测量值与计算值,剔除严重不符合实际的数据组。
  \item 进行雷达测距实验,目标同学在不同位置处小幅度前后晃动。
  \item 负责计算机操控的同学听从指挥点击记录,保存雷达录取数据,对文件重新命
    名。
\end{enumerate}

\subsection{数据记录}%
\label{sub:\arabic{chapter}record}

数据记录见AD\_Data.txt。以汽车经过的数据为例。

\section{实验分析}%
\label{sec:\arabic{chapter}analysis}

如图~\ref{fig:beat},在该锯齿波线性调频连续波雷达中,单通道结构比双通道结构更
为简单,易于实现大时宽带宽积信号的处理。

\begin{figure}[htbp]
  \centering
  \begin{subfigure}[htbp]{0.45\linewidth}
    \centering
    \includegraphics[
      width = \linewidth,
    ]{beat/iq}
    \caption{I、Q通道}%
    \label{fig:beat/iq}
  \end{subfigure}
  \quad
  \begin{subfigure}[htbp]{0.45\linewidth}
    \centering
    \includegraphics[
      width = \linewidth,
    ]{beat/dc}
    \caption{差拍信号去直流}%
    \label{fig:beat/dc}
  \end{subfigure}
  \caption{差拍信号波形}%
  \label{fig:beat}
\end{figure}

如图~\ref{fig:mti},杂波对消前噪声干扰较多,当我们采用相参处理,使各杂波相位
相反,完成杂波对消后,可以有效地去除噪声干扰,提高信噪比,有利于目标信息的提
取。

\begin{figure}[htbp]
  \centering
  \begin{subfigure}[htbp]{0.45\linewidth}
    \centering
    \includegraphics[
      width = \linewidth,
    ]{mti/before}
    \caption{杂波对消前}%
    \label{fig:mti/before}
  \end{subfigure}
  \quad
  \begin{subfigure}[htbp]{0.45\linewidth}
    \centering
    \includegraphics[
      width = \linewidth,
    ]{mti/after}
    \caption{杂波对消后}%
    \label{fig:mti/after}
  \end{subfigure}
  \caption{动目标识别}%
  \label{fig:mti}
\end{figure}

如图~\ref{fig:doppler},雷达对信号匹配滤波完成脉冲压缩,杂波对消后,再进行
MTD,即可得到差拍频率-多普勒频率二维空间输出强度波形图。

\begin{figure}[htbp]
  \centering
  \begin{subfigure}[htbp]{0.45\linewidth}
    \centering
    \includegraphics[
      width = \linewidth,
    ]{doppler/3d}
    \caption{3维}%
    \label{fig:doppler/3d}
  \end{subfigure}
  \quad
  \begin{subfigure}[htbp]{0.45\linewidth}
    \centering
    \includegraphics[
      width = \linewidth,
    ]{doppler/2d}
    \caption{2维}%
    \label{fig:doppler/2d}
  \end{subfigure}
  \caption{差拍-多普勒输出}%
  \label{fig:doppler}
\end{figure}

如图~\ref{fig:mtd},差拍频率对应延时,通过计算即可得到目标与雷达间距离。

\begin{figure}[htbp]
  \centering
  \begin{subfigure}[htbp]{0.45\linewidth}
    \centering
    \includegraphics[
      width = \linewidth,
    ]{mtd/3d}
    \caption{3维}%
    \label{fig:mtd/3d}
  \end{subfigure}
  \quad
  \begin{subfigure}[htbp]{0.45\linewidth}
    \centering
    \includegraphics[
      width = \linewidth,
    ]{mtd/2d}
    \caption{2维}%
    \label{fig:mtd/2d}
  \end{subfigure}
  \caption{动目标检测}%
  \label{fig:mtd}
\end{figure}

最终实验结果如表~\ref{tab:result}。

\begin{table}[htbp]
  \centering
  \caption{实验结果}%
  \label{tab:result}
  \csvautobooktabular[respect percent]{tab/result.csv}
\end{table}

\section{实验讨论}%
\label{sec:\arabic{chapter}discuss}

该实验实质是利用雷达测得的数据进行雷达仿真。

值得注意的是真正的雷达与该仿真算法相同但实现手段不一样。仿真是在通用计算机上
通过通用的科学数据计算软件计算结果,但实际的雷达一般是利用专用的大规模集成电
路或可编程逻辑门阵列依靠硬件的方式实现与软件等效的结果。其好处在与高效性。例
如实际雷达是实时输出结果的,而软件仿真需要等待较长时间。

\section{实验小结}%
\label{sec:\arabic{chapter}conclusion}

为了方便结果的输出对原 matlab 代码进行了修改,主要目的是为了实现结果的自动化
输出,减轻论文插图插表的负担。包括:

\begin{itemize}
  \item 图片自动保存到指定文件夹下和自动插入到论文指定位置。通过在 matlab
    外用一个 shell 脚本新建文件夹,调用 matlab 输出图片,修改后再删除中间文件
    。原先的程序是8张图片在4个窗口中显示出来后用户需要手动截屏或鼠标点击菜单
    栏的图片导出;
  \item 实验结果自动保存到指定文件夹下。原先和图片输出在一起,显示在窗口里;
  \item 对程序进行了格式的规范化。
\end{itemize}

包括之前实验的绘图程序在内的所有程序一并列出在附录~\ref{cha:code}。

\end{document}

