\documentclass[../main]{subfiles}
\begin{document}

\begin{acknowledgement}

  我首先要感谢的是我的指导老师
  \href{http://gsmis.njust.edu.cn/Open/TutorInfo.aspx?dsbh=AXn0dhWNfYeOvvNLjU77xw==&yxsh=9T4MAi3dYTw=}{%
  许志勇}老师,许老师严谨的研究态度深深的影响着我。许老师不仅以严谨的治学态度
  给我以后的学习研究树立了典范,更使我对人生有了新的领悟。虽然我以后可能不会
  从事雷达设计相关方面的工作,但我仍然衷心地感谢许老师的悉心指导!

  感谢我实验的所有搭档和同学!在如此炎热的天气孜孜不倦做实验,对肉体和精神都
  是严峻的考验。能坚持下来,证明了大家的毅力。祝愿能在未来取得更大的成就!

  也感谢一下学校的
  \href{http://bysj.njust.edu.cn/shownews.aspx?newsno=vfRf1W708GGB4ZkVqKVqJw....}{%
  教务处}提供这份实验报告的模板,并对此做一些说明。往届学长学姐的实验报告均缺
  少了声明、英文摘要、关键词、参考文献等。后三项缺少可能影响不大,但我以为缺
  少作者授权本校有权保存、借阅、公开本文的声明从版权意义上是不妥的,所以特意
  按学校要求添加了声明页,所以和老师一开始的模板有些许差别,

  他山之玉,可以攻石。本文开源于
  \href{https://github.com/Freed-Wu/radar-principles-report}{Freed-Wu/radar-principles-report},
  一方面欢迎留言指正,另一方面也希望能为后来者留下前进的足迹!
\end{acknowledgement}

\end{document}

